%
% 6.006 problem set 1
%
\documentclass[12pt,twoside]{article}

\usepackage{amsmath}

\input{macros}

\setlength{\oddsidemargin}{0pt}
\setlength{\evensidemargin}{0pt}
\setlength{\textwidth}{6.5in}
\setlength{\topmargin}{0in}
\setlength{\textheight}{8.5in}

% Fill these in!
\newcommand{\theproblemsetnum}{1}
\newcommand{\releasedate}{September 18, 2016}
\newcommand{\partaduedate}{Thursday, September 15}
\newcommand{\tabUnit}{3ex}
\newcommand{\tabT}{\hspace*{\tabUnit}}

\begin{document}

\handout{Randomized Algorithm for Checking Associativity of Binary Operation}{\releasedate}

\newif\ifsolution
\solutiontrue
\newcommand{\solution}{\textbf{Your Solution:}}

This paper described a randomized algorithm for checking whether a given binary operation has associative property. The solution will prove that the algorithm run in $O(n^2 log(\frac{1}{\delta}))$ time-complexity where n is the number of elements in the binary operation, and $\delta$ is the probality that the algorithm will give the wrong answer.

\setlength{\parindent}{0pt}

\medskip

\hrulefill
\\
\textbf{Lemma 1.} Given set $S$ and binary operation $(C, \circ)$. We define set $G$ as follows. The element is $G$ are sums of elements of $S$ with coefficients is either $0$ or $1$. (i.e, $g \in G \Leftrightarrow g = \sum_{s \in S} \alpha_s s$ where  $\alpha_s \in \{0, 1\} \forall s \in S$). G is equipped with the following operatation.
\\
  $(1)$ Addition: $\sum_{s} \alpha_s s + \sum_{s} \beta_s s = \sum_{s} (\alpha_s + \beta_s)s $ (where $1 + 1 = 0; 0 + 0 = 0; 1 + 0 = 1 + 0 = 1$)
  $(2)$ The operation $\circ$ : $(\sum_{s}\alpha_ss)\circ (\sum_{s}\beta_ss) = \sum_{r}\sum_{s}\alpha_s\beta_s(r \circ s)$
\\
Easily observe that $(C, \circ)$ is associative $\Leftrightarrow (G, \circ)$ is also associative
\\ \\
\textbf{Lemma 2.} We can easily see that there are $2^n$ elements in set $G$. Therefore, checking for associativity require us to check for $8^n$ operations of 3 elements in $G$. We are going to prove that if the operation $\circ$ in $G$ is not associative, then more than $\frac{1}{8}$ of 3-element operation is not associative.
\textbf{Proof.}
\\
Since $\circ$ is non-associative, there exists $a, b, c \in G$ ($a, b, c \neq 0$), such that
$$a \circ (b \circ c) \neq (a \circ b) \circ c$$
By definition, we have $a = \sum_{i = 1}^{n} \alpha_i s$. Since $a \neq 0$, exists $\alpha_m = 1$. We define set 
$$A = \{x=\sum_{i = 1}^{n} \alpha_i s | x \in G, \alpha_m = 0 \}$$
For every $x \in A$, we consider $(a + x) \circ (b \circ c)$ and $((a+x) \circ b) \circ c$.
If they don't equal, we have another 3-element operation that disatifies associativity. If they equal, then since $a, b, c$ disatifies associativity then $x, a, b$ also disatifies associativity.
\\
We can easily observe that $(a + x_i) \neq x_j \forall x_i, x_j \in A$ since $\alpha_m = 1$ in $a$, and $(a + x_i) \neq (a + x_j) \forall x_i, x_j \in A, x_i \neq x_j$ \\
We also have $|A| = 2^{n-1}$, therefore, there are another $2^{n-1}$ set that disatifies associativity.
Prove similarly with $b$ and $c$, we have $8^{n-1}$ set that disatifies associativity
\\
\\
\textbf{Algorithm.}
\begin{enumerate}
\item Choosing randomly 3 element in $G$, from lemma 1, if they dissatify associativity then $S$ is also non-associative. From lemma 2, there is $\frac{1}{8}$ chance for dectecting non-associativity. Checking for associativity will take $O(n^2)$ time-complexity for a brute force solution
\item If we want the error to be an arbitrary $\delta > 0$, repeat step 1 $\log_{8/7}\frac{1}{\delta}$. Therefore, the algorithm runs in $O(n^2 log(\frac{1}{\delta}))$. Choosing $\delta = \frac{1}{n}$, we have an algorithm that runs in $O(n^2 log n)$ time-complexity
\end{enumerate}
In conclusion, if there procedure results in non-associativity, then the outcome is guaranteed since it finds a set that dissatifies associativity. If procedure results in associativity, then the probability of error is $1/n$ 
\
\end{document}
